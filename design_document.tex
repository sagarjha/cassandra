\documentclass {article}

\title {CS 451 Project \\Implementation of Cassandra}
\author {Sagar Jha - 110100024\\ Rohan Gyani - 110040001}

\begin{document}
\maketitle

We describe the functionality of the Cassandra system and indicate the subsets of feature we will implement, along with.

There are four modules that make up the Cassandra process on a single machine : Partitioning module, cluster membership, failure detection module and storage engine module.

Event driven substrate SEDA architecture

Data Model :
key value. value highly structured. atomic operation under a single row key. columns are grouped under column families. columns time sorted or name sorted.

API :
insert, get and delete

Partitioning :
Consistent hashing (similar to Chord's)

Replication :
Coordinator manages replication. Options are Rack Unaware, Rack Aware and Datacenter Aware. We implement Rack Unaware. Replication factor is N, then the N-1 nodes next to coordinator hold the replicas.

Read/Write Requests : 
1. Identify the node(s) that own the data for the key
2. Route the requests to the nodes and wait on the responses to arrive
3. If timeout, fail the request and return to client
4. Schedule a repair of the data at any replica if needed
 Features - Implement Asynchronous writes, Synchronous writes (Optional)
\end{document}