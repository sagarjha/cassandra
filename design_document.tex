\documentclass {article}

\title {CS 451 Project \\Implementation of Cassandra}
\author {Sagar Jha - 110100024\\ Rohan Gyani - 110040001 \\ Abhishek Gupta - 110040067}

\begin{document}
\maketitle

\section {Problem Description}
We implement a key value store with two operations : insert and get.

Insert : Takes as input table name, key and value. Value is in the form supercolumn : column, where we identify supercolumns as a group of columns.
Get : Takes as input table name, key and supercolumn name. Returns the set of values (the columns)

We also implement a messaging application that makes use of the cassandra system to store messages. It offers two functionalities other than send.

Term Search : Search for a keyword. Returns all messages that contains the keyword in time-sorted order.

Interaction Search : Search for interactions with a user. Returns all messages that were sent to this user in time-sorted order.

We use the messaging application for performance and scalability testing.

Performance Testing : Populate the table with messages and then compute the average time taken for insert and get.

Scalability Testing : We control the number of users and the amount of interactions between them and check the variability in average times for insert and get.

\section {System Design}
Storage at servers : For a given table, key and supercolumn, we have a unique file on disc, path of which is <table>/<key>/<supercolumn>. Each such file stores all the values. We write sequentially to the file and hence the data is sorted by time of write.

We store the records in memory till a certain number of records before flushing to disc.

On disc the format for storage is $<column-name> <size> <column-name> <size> ...$

In memory, the data structure is an arrayList of keys and for every key, we have a map of supercolumns with the list of columns.

Key-space partitioning : Chord's algorithm is used for partitioning and replication. If a key falls into the domain of a certain node, that node is called the coordinator of the key. If $k$ is the replication factor, the coordinator is responsible for storing it in its $k-1$ successors.

On an insert, we first direct the operation to the coordinator of the key. The coordinator sends insert commands to all the $k-1$ successors also. We perform synchronous writes and acknowledgement is send only after all the $k$ replicas are stored.

On a get, we check if the node is one of the holders of the key. If not, then we direct the operation to the coordinator of the node. So, a get operation uses only one of the nodes that has the data.

\section {Environment for execution}
We require JVM. We use RMI heavily for communication. We will test the system on Linux Ubuntu.

\section {Performance and Scalability (Optional)}
This is just as desribed in the first section. We are not sure, if we will be able to perfomance these tests.

\end{document}